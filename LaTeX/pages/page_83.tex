\begin{itemize}
	\renewcommand{\labelitemi}{\scriptsize$\blacksquare$}
	\item Obtain commitments and customer approvals needed for the timely accomplishment of the project.
	\item Joint Management Reviews that normally apply to	software are:

	\begin{itemize}
		\item Program Quarterly Status Reviews.
		\item Monthly Status Reviews.
		\item Process Compliance Audits.

	\end{itemize}
\end{itemize}


As the SPM, you will be an intimately involved with Joint
Management Reviews along with your CSwE and appropriate members of the SwIPTs. All participants support JMRs
by providing progress-to-date and overviews of the software
products. Identification of the JMRs that apply, the schedule
for each, the process to be followed for each, the documents
to be reviewed, and the personnel involved should be defined
in the SDP or in a \textit{Software Reviews Plan} as an addendum to
the SDP.

A guideline to the specific software documentation product reviews can be described in a table in the SDP similar
to Table 5.7. It is an example of the \textit{evolution of document
maturity} from draft versions to preliminary versions and
then to baselined versions followed by updates as needed.
The example in Table 5.7 maps software documentation to
formal software reviews and is complementary to Table 13.1
that relates the development of software documents to software development activities.


\textbf{Lessons Learned.} At one time I was the Software
Lead on the Space Station Freedom program
at Lockheed. Since we were a subcontractor,
every month the VP in charge of our program
participated in a JMR at the prime contractor’s
site. Since his presentation was to their Senior
Managers it had a time constraint; each area of
our responsibility was restricted to a one chart
update. There was so much software activity
going on that it was difficult to get a meaningful
update on one chart. So, I came up with a technique of multiplying the content density of that
one chart. I did that by using the four corners of
each rectangular box on the chart to contain, or
represent, some data. It was a simple but effective
way of conveying more of the key information in
the same limited space.


However, I used only three of the four corners. After our VP finished my chart, the prime
contractor’s VP asked, “Is that a Gechman
chart?” The answer was “yes,” and the prime’s VP
said: “Tell Marv that we are upset he did not use
all four corners!” This humorous reaction was
an example of the rapport I was able to establish
with our customer; it made my job much easier
and much more enjoyable. I highly recommend that you always try to do the same with your
customer. Also, be creative in preparing briefing
materials and try to get the top level on one page
as discussed in Subsection 1.4.7.

\section{Software Development Life Cycle Process}

Critical strategies must be identified in the SDP to ensure
that software groups provide additional oversight and focus
on incorporating critical requirements into the SIs. There are
almost always some key software requirements that are critical
cornerstones such as for safety, security, human factors, privacy
protection, reliability, maintainability, availability, performance,
etc. Software Project Managers must develop and employ strategies to ensure that these critical requirements are satisfied.

The critical strategies should be documented in the SDP.
It should include both test and analyses activities to ensure
that the requirements, design, implementation and operating procedures, for the identified computer hardware and/
or software, \textit{minimize or eliminate the potential} for violating
established risk mitigation strategies. The SDP should also
indicate how evidence is to be collected to prove that the
assurance strategies have been successful.

\subsection{Software Safety}

Software safety requirements involve Software Items or
Software Units (SUs) whose failure may result in a system
condition that can cause death, injury, occupational illness,
damage to or loss of equipment or property, or damage to the
environment. Each software-related Safety Critical requirement identified must be documented in the safety requirements Section of the Software Requirements Specification
(SRS) and identified by a unique product identifier

If aviation safety standards are specified in the contract as compliance documents, your SDP or an addendum
to it must describe the approach for complying with those
standards and regulations. There are other software safety
standards that you may be required to adhere to in specific
industries such as transportation, aerospace, nuclear power
plants and national defense. Safety and reliability issues also
occur in normal every day safety-related scenarios such as the
software that controls elevators, medical databases, microwave ovens, and many more.

The activities required for ensuring that Safety Critical
software requirements are met for the program must be shared
between the System Safety group, at the program level, and
the Subsystem Software Team. Each SwIPT should assign
responsibilities for safety issues and for coordination with
System Safety. The Software Team is responsible for developing system software that is safe to operate and compliant
with all appropriate safety standards and requirements.