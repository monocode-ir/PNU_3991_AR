\onecolumn

\begin{table*}[t] % t for top also possible <===========================
	\caption{A table displaying the sequences for the strands which make up the DNA receptors for both the bulk and vesicle experiments} 
	\label{sequences}\centering
	\centering
	\begin{tabular}[t]{| c | c | c | c | c | c | c | c | c | c |}
		\hline  Software Document
		& SFR & SAR & PDR & CDR & IRR   & SI TRR & PTR or BTR & FAT TRR & SQT TRR\\
		\hline \\ 
		\hline Software Development Plan (SDP)  & &  B & U & U & U &  &  &  &  \\
		\hline Software Metrics Report   & P &  B & U & U & U & U  & U & U &  \\
		\hline Software Master Build Plan (SMBP)   &  &  D & P & B &  &  &  &  &  \\
		\hline Software Requirements Specification
		(SRS) & & P &  B & U & U &  &  &  & \\
		\hline Interface Requirements Specification
		(IRS) & & P &  B & U & U &  &  &  & \\
		\hline Interface Control Document (IFCD) & & B &  U &  &  &  &  &  & \\
		\hline Software Design Description (SDD)  & &  &  P &  B &  &  &  &  & \\
		\hline Software Architecture Description (SAD) & &  &  P &  B &  &  &  &  & \\
		\hline Software Test Plan (STP) & &  &  P &  B & U & U & U &  & \\
		\hline Interface Design Document (IDD)  & &  &  D/P &  B & U &  &  &  & \\
		\hline Database Design Description (DBDD)  & &  &  D/P &  B & U &  &  &  & \\
		\hline Software Installation Plan (SIP) & &  &   & D  &  &  & P &  & \\
		\hline Software Transition Plan (STrP) & &  &   & D  &  &  & P &  & \\
		\hline Software User Manual (SUM)  & &  &   & D  &  &  & P &  & \\
		\hline Firmware Support Manual (FSW) & &  &   &  &  D &  & P &  & B\\
		\hline Computer Programming Manual (CPM) & &  &   &  &  D &  & P &  & B\\
		\hline Software Test Description (STD)  & &  &   &  &   & D/P & B & U & \\
		\hline Software Test Report (STR)   & &  &   &  &   &  & B &  & \\
		\hline Software Version Description (SVD) & &  &   &  &   & D & P & B & U\\
		\hline Software Product Specification (SPS)  & &  &   &  &   & D & P & B & U\\
		\hline
	\end{tabular}
\end{table*} % <========================================================

\begin{table}
	\begin{center}
		\caption{\textbf{Software Documentation Maturity Mapped to Software Reviews}}
		\label{Table 5.7}
		\centering
		\begin{tabular}{| m{4.8cm} | m{0.7cm}  | m{0.7cm}  | m{0.7cm}  | m{0.7cm}  | m{0.7cm}  | m{1.2cm}  | m{1.2cm}  | m{1.4cm} | m{1cm} | }
 	\hline  Software Document
 & SFR & SAR & PDR & CDR & IRR   & SI TRR & PTR or BTR & FAT TRR & SQT TRR\\
			
			\hline Software Development Plan (SDP)  & &  B & U & U & U &  &  &  &  \\
			\hline Software Metrics Report   & P &  B & U & U & U & U  & U & U &  \\
			\hline Software Master Build Plan (SMBP)   &  &  D & P & B &  &  &  &  &  \\
			\hline Software Requirements Specification
			(SRS) & & P &  B & U & U &  &  &  & \\
			\hline Interface Requirements Specification
			(IRS) & & P &  B & U & U &  &  &  & \\
			\hline Interface Control Document (IFCD) & & B &  U &  &  &  &  &  & \\
			\hline Software Design Description (SDD)  & &  &  P &  B &  &  &  &  & \\
			\hline Software Architecture Description (SAD) & &  &  P &  B &  &  &  &  & \\
			\hline Software Test Plan (STP) & &  &  P &  B & U & U & U &  & \\
			\hline Interface Design Document (IDD)  & &  &  D/P &  B & U &  &  &  & \\
			\hline Database Design Description (DBDD)  & &  &  D/P &  B & U &  &  &  & \\
			\hline Software Installation Plan (SIP) & &  &   & D  &  &  & P &  & \\
			\hline Software Transition Plan (STrP) & &  &   & D  &  &  & P &  & \\
			\hline Software User Manual (SUM)  & &  &   & D  &  &  & P &  & \\
			\hline Firmware Support Manual (FSW) & &  &   &  &  D &  & P &  & B\\
			\hline Computer Programming Manual (CPM) & &  &   &  &  D &  & P &  & B\\
			\hline Software Test Description (STD)  & &  &   &  &   & D/P & B & U & \\
			\hline Software Test Report (STR)   & &  &   &  &   &  & B &  & \\
			\hline Software Version Description (SVD) & &  &   &  &   & D & P & B & U\\
			\hline Software Product Specification (SPS)  & &  &   &  &   & D & P & B & U\\
			\hline
		\end{tabular}
	\end{center}
\end{table}


\textbf{D = Draft; P = Preliminary; B = Baselined; U = Updated (As Required);} \textbf{SFR} = = System Functional Review; \textbf{SQT TRR} = System Qualification Test—Test Readiness Review; \textbf{SAR} = Software Requirement and Architecture Review; \textbf{IRR} = Integration Readiness Review; \textbf{PDR} = Preliminary Design Review; \textbf{SI TRR}  = Software Item—Test Readiness Review; \textbf{CDR} = Critical Design
Review;  \textbf{PTR or BTR} R = Post-Test Review or Build Turnover Review; \textbf{FATTRR} = Factory (or Element) Qualification (or Acceptance) Test—Test Readiness Review.



The general approach to managing software Safety
Critical development activities for the program should be
to \textit{integrate safety management into the Software Life Cycle
activities}. System Safety should play an integrated role in the
software development process providing the System Safety
group with visibility into the software development activities
that are critical to program safety issues, as well as providing the SwIPTs with the input required to ensure that safety
issues are addressed effectively. Details regarding software safety should be included in a \textit{System Safety Program Plan} or equivalent.

The SDP should require Software Safety Engineers to
\textit{define classifications for Safety Critical SIs and SUs}. All SIs and
SUs should be categorized according to these Safety Critical
classifications. To prepare these classification levels, consideration should be given to: the severity and probability
of hazards the SIs or SUs may contribute to (as determined
by a hazards analysis); the potential for the SIs or SUs to