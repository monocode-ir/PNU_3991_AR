\textbf{Software Dependability}. Dependability is the sum
result of effective strategies for reliability, maintainability
and availability (RMA) and the SDP should describe the
overall approach to develop these strategies. Software RMA
practices must be incorporated throughout all software
development activities; they provide the building blocks for
dependability. Effective strategies for RMA also helps to
ensure that software meets system requirements with minimum risks, maintains the integrity of the Software Design,
and minimizes life cycle costs.

\textbf{Software Reliability.} Software reliability models should
be used to assist in making predictions about the software
system expected failure rates. The reliability tasks must be
integrated with Quality Assurance, Product Evaluations,
Maintainability, and other Engineering activities to avoid
duplication and provide a cost-effective program. Software
reliability should involve detection, reporting, quantification, and correction of software deficiencies throughout all
design, development and testing activities.

\textbf{Software Maintainability.} There are two major aspects
of software maintainability: software \textit{restorability} and software \textit{reparability}:

\begin{itemize}
	\renewcommand{\labelitemi}{\scriptsize$\blacksquare$}
	\item Software restorability is defined as the process of
	restoring the software to an operational state after the
	occurrence of hardware or software failures. Ineffective
	software restorability can be a large contributor to downtime and thus can significantly affect system availability.
	The need for rapid software restorability is a major driver
	of the software architecture and design task.
	\item Development of maintainable software, from a software reparability perspective, involves planning and
	establishing the software development methodology,
	environment, standards, and processes with an objective of making Software Maintenance changes efficiently and effectively (e.g., not requiring an engine
	disassembly to change the oil).

\end{itemize}

Some methodologies, such as Object-Oriented Design,
development and programming, may produce softwarerelated products that are more maintainable than other
approaches. The design must be captured and retained in
the Software Engineering tools and subject to Configuration
Management (CM) processes. Similarly, the software CM
tools provide support to Software Maintenance needs. Other
tactics that can be described in the SDP to improve maintainability may include:

\begin{itemize}
	\renewcommand{\labelitemi}{\scriptsize$\blacksquare$}
	\item The Software Development Environment (SDE),
	covered in Section 9.1, must be sized to include sufficient capacity to support post-deployment Software
	Support requirements, thus promoting long-term
	maintainability.
	\item Software standards must be established for each programming language to ensure that consistent programming styles are applied by all Developers and that the
	software and supporting documentation are complete
	and understandable.
	\item The Software Product Evaluations should assess compliance with the standards to ensure that they are consistently applied.
\end{itemize}

\textbf{System Availability.} A high availability rate for access
to the system is the by-product of effective RMA practices
as well as accurate estimation of user needs. By performing
modeling and trend analysis, based on historical trend data
and collected metrics, software reliability and availability
can be predicted and the necessary Corrective Actions taken
to achieve RMA and system requirements.

\subsection{Assurance of Other System Critical requirements}

Critical software requirements should be tracked and monitored throughout all the software development activities just
like other software requirements. However, in addition to
the standard testing and quality assurance procedures for
other software requirements, the SwIPT should follow an
assurance strategy designed to ensure that hazardous or compromised conditions are eliminated or minimized for each
development activity. This strategy should be to:

\begin{itemize}
	\renewcommand{\labelitemi}{\scriptsize$\blacksquare$}
	\item Identify and document critical requirements in the
	appropriate SRS sections.
	\item Document the specific Software Units that contribute
	to these critical requirement.
	\item Define specific SI testing procedures that execute all
	affected SUs to determine compliance.
	\item Execute the security and privacy testing procedures
	at each SI build when affected Security Critical and
	Privacy Critical SUs have changed.
	\item Execute the safety-related test cases at each SI build for
	SIs with Safety Critical SUs, even if the units have not
	changed.
	\item Update safety analysis, models and modeling results at
	any time as required.
\end{itemize}

The CSwE and SQA should review the procedures followed by the SwIPT and the products produced for critical requirements compliance as part of the normal reviews
of each development activity. The CSwE should focus on
identifying evidence that the general strategy stated above
is being implemented. SQA should evaluate the process of
performing the critical requirements testing, the successful
completion of the testing, and the proper documentation of
the results.