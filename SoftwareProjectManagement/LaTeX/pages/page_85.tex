provide Safety Critical monitoring or mitigation actions; and
how the SIs or SUs handle and protect Safety Critical data.
System and Software Safety Engineers should:

\begin{itemize}
	\renewcommand{\labelitemi}{\scriptsize$\blacksquare$}
	\item Participate in system and software requirements analysis to generate additional functional or performance
	requirements to assure safe operations and safety contingency actions.
	\item Monitor these additional software requirements to
	assure they are properly specified and traced to documented Safety Critical hazards.
	\item Assure that unsafe operations are not specified by existing requirements.
	\item Participate in design reviews to prevent unsafe
	approaches from being applied.
	\item Track internal and external safety-related interfaces to
	assure they are fully documented and unambiguous.
	\item Participate in the review of test procedures to assure Safety
	Critical requirements are properly interpreted and tested.
	\item Participate in the evaluation of Safety Critical code
	changes and review regression tests.
	\item Document Safety Critical criteria used in selecting
	COTS, GOTS and reused code.
\end{itemize}

\subsection{Software Security}
Software security (renamed Information Assurance) involves
SIs and SUs whose failure may lead to a breach of system
security or a compromise of proprietary or government classified data. Each software-related Security Critical requirement identified should be documented in the security and
privacy protection requirements section of the SRS. Security
requirements can be derived from the System Specification. 

\textit{Security concerns can have a significant impact on
your software architecture.}

If applicable, security services provided by the program
for all projects should be documented in an Information
Assurance (IA) Plan and should provide “layers” of structured defense from commercial packages (such as antivirus software and firewalls) to elaborate National Security
Agency (NSA) approved Type-1 encryption algorithms. The
SDP should state that software portions subject to security
product certification and accreditation must be developed in
accordance with the Security Plan.

Software security requirements should be flowed down
to subcontractors with the normal requirements analysis
process. The Software Design activity must conform to the
security architecture as described in the IA Plan. Also, when
developing the software schedules, and the build plan, the
security certification and accreditation need dates must be
accounted for.

\subsection{Human Systems Integration}

\textit{A system (or product) is ultimately judged good or
bad by its end-users.}

\textit{Human System Integration} (HSI), also called the HumanComputer Interface, is a disciplined and interactive Systems
Engineering approach for integrating human considerations
into System Development, Design and Life Cycle management. Applying HSI techniques improves total system
performance and can reduce human errors and the cost of
operations across the system’s life cycle. Significant improvements in performance can be made with design features that
often cost almost nothing to implement. \textit{How people perform
with technology is a critical component of total system performance.} Human performance can be substantially improved,
and the likelihood of errors reduced, simply by designing
a system that is compatible with the characteristics of the
people who operate and maintain it.

If good requirements are generated, the system will
support users in performing their tasks effectively and efficiently. This can be achieved with the support of trained HSI
Engineers who provide the applied technical expertise. One
of the first Systems Engineering objectives of a project is to
define the “who, what, when, where, and why” of the system
being designed and built. That normally takes place during
the Concept Development phase. At that early phase of the
life cycle, inputs from potential users may not be available,
or as useful, as inputs from the HSI Engineers because the
system analysis they can conduct enables them to understand the role of the human in the system and the specific
tasks people perform during the required human system
interaction.

The Department of Defense (DoD) has mandated inclusion of HSI in the development of military systems. DoD
Instruction (DoDI) 5000.02, Enclosure 7 addresses HSI,
stating that the Program Manager should plan for and effect
HSI, beginning early in the acquisition process and throughout the product life cycle. It charges the Program Manager
with the responsibility for ensuring HSI is considered at each
program milestone. The design aspects of HSI are discussed
in Subsection 8.4.2.

Some Program/Project Managers do not fully appreciate
the ways in which HSI can improve system performance, or
they remain confused about how to effectively incorporate
HSI into their programs. This is due to a number of fundamental gaps in understanding what HSI can do. As the SPM,
you should not subscribe to the following seven HSI myths
identified by (Endsley, 2016):

\begin{itemize}
	\renewcommand{\labelitemi}{\scriptsize$\blacksquare$}
	\item \textit{Myth 1}: HSI means asking users what they want: User
	input is an important element of good System Design,
	but relying solely on their input is insufficient evidence of good HSI since users often miss the subtle features
	of technology that can negatively impact human
	performance.

